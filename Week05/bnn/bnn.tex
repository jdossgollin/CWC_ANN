\documentclass{beamer}
%
% Choose how your presentation looks.
%
% For more themes, color themes and font themes, see:
% http://deic.uab.es/~iblanes/beamer_gallery/index_by_theme.html
%
\mode<presentation>
{
  \usetheme{default}      % or try Darmstadt, Madrid, Warsaw, ...
  \usecolortheme{default} % or try albatross, beaver, crane, ...
  \usefonttheme{default}  % or try serif, structurebold, ...
  \setbeamertemplate{navigation symbols}{}
  \setbeamertemplate{caption}[numbered]
} 

\usepackage[english]{babel}
 % \usepackage[utf8x]{inputenc}
\usepackage[style=authoryear]{biblatex}
\addbibresource{references.bib}

\title[Your Short Title]{Bayes Networks and Probalistic Programming}
\author{Adam Massmann}
\institute{Water Center NN Meetings}
\date{Week 5: Oct. 17th, 2017}

\begin{document}

\begin{frame}
  \titlepage
\end{frame}

% Uncomment these lines for an automatically generated outline.
%\begin{frame}{Outline}
%  \tableofcontents
%\end{frame}

\section{Introduction - Information Theory}

\begin{frame}{Introduction - Information Theory (see section 1.6 in Bishop)}
% \vskip 1cm
\begin{itemize}
\item Consider the amount of information gained/learned by an event or observation (we'll call $x$). We would receive more new information from a very surprising event than an event we expect (because we already know something about the expected event).
\item So if we want to quantify this ``amount of information'' contained in an event we should use a function of the probability of the event ($p(x)$). The amount of information we'll call a function $h(\cdot )$, which will be a function of $p(x)$.
\end{itemize}

\end{frame}


\begin{frame}{Guidance for the functional from of $h(\cdot ) $}
% \vskip 1cm
\begin{itemize}
\item If two events $x$ and $y$ are independent, then the amount of information gained by both events should be $h(x,y) = h(x) + h(y)$.
\item We also know that the joint probability of $x$ and $y$'s occurrence would be: $p(x,y) = p(x) \; p(y)$.
  \item So the question is, what function $\hat{h}$ satisfies: $h(x,y) = \hat{h}(p(x)\; p(y)) = \hat{h}(p(x)) + \hat{h}(p(y))$?
\end{itemize}

\end{frame}



\begin{frame}{Information Entropy}
% \vskip 1cm
\begin{itemize}
\item $\hat{h}(\cdot ) = \log ( \cdot )$ satisfies $h(x,y) = \hat{h}(p(x)\; p(y)) = \hat{h}(p(x)) + \hat{h}(p(y))$, so $h(\cdot ) = \log(p(\cdot ))$.
\item It's desirable for $h$ to be positive, so because $0 \leq p \leq 1$, lets make it $h(\cdot ) = -\log p(\cdot )$.
\item Now say we have a bunch of random variables $x$ for which we want to know the average amount of information (i.e. expectation of $h(x)$). This would be given by:
  \[H[x] = - \sum_x p(x) \log p(x) \]
\item This is known as the \textit{entropy} of $x$.
\item Extending this to continuous variables gives the \textit{differential entropy}:
  \[H[x] = - \int p(x)\, \log p(x)\, dx\][\cite{bishop}, \cite{shannon1948}]
\end{itemize}
\end{frame}

\begin{frame}{So what does information entropy look like?}
    % \vskip 1cm
\begin{itemize}
\item From thermodynamics and statistical mechanics we have some idea of entropy as a measure of the disorder or randomness in a system. For information theory it is similar.\footnote{von Neumann told Shannon he should also call it entropy because ``nobody knows what entropy really is, so in any discussion you will always have an advantage.''}
\end{itemize}

\begin{figure}
\includegraphics[width=3in]{entropy.png}
\caption{\label{fig:entropy}From \cite{bishop}: Histograms of two probability distributions over thirty bins illustrating the higher value of entropy H for the broader distribution. The largest entropy would arise from a uniform distribution that would give $H = -\ln 1/30 = 3.40$}
\end{figure}

\end{frame}


\section{Kullback-Leibler divergence}
\begin{frame}{Kullback-Leibler divergence}
\begin{itemize}
\item Why should we even care about information theory or entropy?
  \begin{itemize}
  \item Because we can use entropy ideas to perfrom inference on a probalistic model (e.g. a Bayes network), given data.
  \item Say we have some phenomenon with a true probability distribution $p(x)$, which we are approximating with some [possibly parametric] distibution $q(x)$.
  \item Then the additional necesary information required to communicate the value of $x$ as consequence of using $q(x)$ would be:
  \end{itemize}
\end{itemize}
\begin{equation}
  \begin{split}
    KL(p||q) & = - \int p(x) \ln q(x) dx \, - \left(-\int p(x) \ln p(x) dx \right) \\
    & =  - \int p(x) \ln  \frac{q(x)}{p(x)}
  \end{split}
\end{equation}
  This is known as relative entropy or  Kullback-Leibler (KL) divergence (\cite{kullback1951}).
\end{frame}

\begin{frame}{Properties of Kullback-Leibler divergence}
  \begin{itemize}
  \item Note that it is not a symmetrical quantity (e.g. $KL(p||q) \ne KL(q||p)$).
  \item Also, $KL(p||q) \ge 0$,
  \item and $KL(p||q) = 0$ only if p and q are identical (see \cite{bishop} for proof).
  \item So practically speaking KL divergence is very useful as a cost function quantifying the similarity between two probability distributions.
  \end{itemize}
\end{frame}

\begin{frame}{Using KL divergence for inference}
  \begin{itemize}
  \item Say we have $N$ observations of data $x_n$ from some unknown probaility distribution $p(x)$.
  \item We want to try to approximate $p(x)$ with a parametric disribution $q(x|\theta)$, by minimizing the KL diverence which can be approximated by:
  \end{itemize}
  \begin{equation}
    KL(p||q) \simeq \frac{1}{N} \sum_{n=1}^{N} \left[ -\ln q(x_n | \theta) + ln(p (x_n))\right]
  \end{equation}
  \begin{itemize}
  \item The second term is not a function of $\theta$, and the first term is just the negative log liklihood. So for this example, minimizing KL-divergence is the same as minimizing the negative log liklihood, which we saw in Week 1!
  \end{itemize}
\end{frame}

\section{References}
\begin{frame}{References}
  \begin{block}{See section 1.6 in Bishop}
    \printbibliography
  \end{block}
\end{frame}

% \begin{table}
% \centering
% \begin{tabular}{l|r}
% Item & Quantity \\\hline
% Widgets & 42 \\
% Gadgets & 13
% \end{tabular}
% \caption{\label{tab:widgets}An example table.}
% \end{table}


\end{document}
